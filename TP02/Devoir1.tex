% filepath: c:\Users\j.noel\Desktop\EPHEC\devIII-tp\TP02\REPORT_TEMPLATE.tex
% Compiler: pdflatex REPORT_TEMPLATE.tex (ou xelatex)
\documentclass[11pt,a4paper]{article}
\usepackage[utf8]{inputenc}
\usepackage[T1]{fontenc}
\usepackage[french]{babel}
\usepackage{geometry}
\usepackage{graphicx}
\usepackage{listings}
\usepackage{xcolor}
\usepackage{hyperref}
\usepackage{amssymb}

\geometry{margin=2.5cm}
\lstset{
  language=TypeScript,
  basicstyle=\ttfamily\small,
  keywordstyle=\color{blue},
  commentstyle=\color{gray},
  breaklines=true
}

\title{T2112 - Devoir 1 \\Rapport}
\author{Nom : Charlier Thomas }

\begin{document}
\maketitle
\tableofcontents
\bigskip

\section*{Résumé}

\textit{Dans ce devoir, j'ai ajouté une page de connexion permettant de vérifier un couple username/password avant d'accéder à la gestion des utilisateurs. La page permet d'ajouter et de supprimer des utilisateurs via un formulaire HTML, en interagissant avec l'API Express et la base de données SQLite.}






\section{Objectifs (ce que je souhaite atteindre)}
\textit{
- Créer une page de connexion avant la gestion des utilisateurs.
- Vérifier le couple username/password via POST /api/login.
- Ajouter de nouveaux utilisateurs dans la base de données avec les champs : nom, prénom, username, password. Sans sécurité sur les password
- Supprimer des utilisateurs via DELETE /api/users/:id.
- Afficher la liste des utilisateurs avec GET /api/users.
- Rediriger l'utilisateur vers la page de gestion uniquement après connexion réussie.
}





\section{Stratégie et raisonnement}
\textit{
- J'ai choisi Express pour créer l'API backend et SQLite via Sequelize pour la base de données, pour garder la simplicité et la portabilité.
- Séparation claire du code : models/user.ts pour le modèle, services/userService.ts pour la logique métier, routes/userRoutes.ts pour les endpoints.
- Frontend minimal avec HTML + JavaScript pour interagir avec l'API via fetch.
- Redirection simple après login pour limiter l'accès à la gestion des utilisateurs.
- Pas de sécurité sur les mots de passe pour ce TP, mais validation côté serveur et côté client.
}





\section{Réalisation (ce que j'ai réellement atteint et comment)}
\subsection{Fonctionnalités ajoutées}
\subsection{Fonctionnalités ajoutées}
\textit{
- Page de login avec vérification du couple username/password.
- Endpoint POST /api/login pour la connexion.
- Endpoint GET /api/users pour lister les utilisateurs.
- Endpoint POST /api/users pour ajouter un utilisateur.
- Endpoint DELETE /api/users/:id pour supprimer un utilisateur.
- Formulaire HTML pour ajouter des utilisateurs et bouton de suppression.
- Redirection vers la page de gestion après login réussi.
- Base de données synchronisée automatiquement avec Sequelize.
}





\subsection{Exemple de code}
\textit{
L'endpoint \texttt{/login} est un endpoint POST qui reçoit en corps de requête les champs \texttt{username} et \texttt{password}. 
La logique est la suivante : 
\begin{enumerate}
  \item Le serveur vérifie que les deux champs ne sont pas vides. Si l'un d'eux est vide, une réponse HTTP 400 est renvoyée avec un message d'erreur.
  \item Si les champs sont renseignés, le serveur recherche dans la base de données un utilisateur correspondant au couple \texttt{username/password}.
  \item Si un utilisateur est trouvé, le serveur renvoie un code HTTP 200 avec un message de confirmation de connexion réussie.
  \item Si aucun utilisateur ne correspond au couple fourni, le serveur renvoie un code HTTP 404 avec un message indiquant "Utilisateur inconnu".
  \item En cas d'erreur interne (ex : problème de base de données), le serveur renvoie un code HTTP 500 avec un message d'erreur générique.
\end{enumerate}
Cette approche permet de gérer correctement la validation des données, la réponse en cas de succès ou d'erreur, et fournit un retour clair au client.
}\begin{lstlisting}[caption=Exemple d'endpoint ajouté]
router.post("/login", async (req: Request, res: Response) => {
  const { username, password } = req.body;
  if (!username || !password) return res.status(400).json({ message: "Nom et mot de passe requis" });

  try {
    const user = await User.findOne({ where: { username, password } });
    if (!user) return res.status(404).json({ message: "Utilisateur inconnu" });
    res.status(200).json({ message: "Connexion réussie" });
  } catch (err) {
    res.status(500).json({ message: "Erreur serveur" });
  }
});
\end{lstlisting}







\subsection{Captures d'écran}
\begin{figure}[h]
  \centering
  \includegraphics[width=0.8\textwidth]{https://github.com/JN-EPHEC/individual-thomas40404/blob/homework_1/TP02/Screen/connexion.png}
  \caption{Page de connexion}
\end{figure}

\begin{figure}[h]
  \centering
  \includegraphics[width=0.8\textwidth]{https://github.com/JN-EPHEC/individual-thomas40404/blob/homework_1/TP02/Screen/USER_JSON.png}
  \caption{Liste des utilisateurs via GET /api/users}
\end{figure}

\begin{figure}[h]
  \centering
  \includegraphics[width=0.8\textwidth]{https://github.com/JN-EPHEC/individual-thomas40404/blob/homework_1/TP02/Screen/db.png}
  \caption{Conception de la DB}
\end{figure}

\begin{figure}[h]
  \centering
  \includegraphics[width=0.8\textwidth]{https://github.com/JN-EPHEC/individual-thomas40404/blob/homework_1/TP02/Screen/delet.png}
  \caption{Suppression d'un utilisateur via DELETE /api/users/:id}
\end{figure}
\begin{figure}[h]
  \centering
  \includegraphics[width=0.8\textwidth]{https://github.com/JN-EPHEC/individual-thomas40404/blob/homework_1/TP02/Screen/usersHTML.png}
  \caption{Page de gestion des utilisateurs}
\end{figure}





\section{Annexes}
\textit{Ajoutez ici toutes les annexes, pour référencer, utilisez la balise ref avec le nom du label associé \ref{ann1}}

\subsection{EXEMPLE : Code source - routes/user.ts}
\label{ann1}



\subsection{EXEMPLE : Screenshot - ABC}
\label{abc}



\end{document}
